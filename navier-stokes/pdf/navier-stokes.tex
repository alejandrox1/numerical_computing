\documentclass[10pt,a4paper,draft]{article}
\usepackage[utf8]{inputenc}
\usepackage{amsmath}
\usepackage{amsfonts}
\usepackage{amssymb}


\begin{document}

\title{Navier-Stokes Equations} 
\author{\texttt{@X1alejandrox3}} 
\date{\today}  
\maketitle


\section{Differential Form}

\subsection{Reynolds Transport Theorem}
Consider an integration of $\mathbf{f} = \mathbf{f}(\mathbf{x},t)$ over the time-dependent region $\Omega (t)$ with boundary $\partial \Omega(t)$.
Taking a derivative with respect to time,
$$
\frac{d}{dt} \int_{\Omega (t)} d\mathbf{V} \, \mathbf{f}
$$

To move the integral inside the integration, we need to account for the time dependence of $\mathbf{f}$ along with the introduction and removal of space from $\Omega$ due to its dynamic boundary.


Reynold's transport theorem allows us to do so!
$$
\frac{d}{d t} \int_{\Omega (t)} d\mathbf{V} \, \mathbf{f} 
= \int_{\Omega (t)} d\mathbf{V} \, \frac{\partial \mathbf{f}}{\partial t}
+ \int_{\partial \Omega (t)} d\mathbf{A} \, \left(\mathbf{v}^b \cdot \mathbf{n}\right) \mathbf{f}
$$

Here, $\mathbf{n}(\mathbf{x},t)$ is the outward-pointing unit normal vector and $\mathbf{v}^b(\mathbf{x},t)$ is the velocity of the area element.


\subsection{Conservation of Mass}
$$
\frac{D M_{sys}}{D t} = 0
$$

Using Reynold's theorem we can then write a system of equations for a control volume (cv).
We will have a couple terms: 
\begin{enumerate}
\item Rate at which mass in the control volume changes.
\item Flux accross control surface (net rate at which mass is flowing).
\end{enumerate}
both of this terms have to negate each other in order for mass to be conserved:

$$
\frac{DM}{Dt} = 
\frac{\partial}{\partial} \int_{cv} \rho \, dV 
+ \int_{cs}  \rho \mathbf{V} \cdot \mathbf{\hat{n}} \, dA
= 0
$$
Where $\mathbf{v}$ is the velocity of the fluid. \\~\\

\textbf{Differential Form:}

Consider a small fluid element of size $\delta x \delta y \delta z$.

Assuming that the density is uniform accross the differential of volume (very small volume), we can see that the rate at which the mass inside the control volume changes is given by,
$$
\frac{\partial \rho}{\partial t} \delta x \delta y \delta z
$$

The $x$ componenet of the mass flow rate, accross a face of an infinitesimal cube:
$$
\left[ 
\rho u 
+ \frac{\partial}{\partial x} \left(\rho u\right) \frac{\delta x}{2}
\right] \delta y \delta z
$$

the flux of mass accross the opposite face:
$$
\left[
\rho u
- \frac{\partial}{\partial x}\left(\rho u\right) \frac{\delta x}{2}
\right] \delta y \delta z
$$

So the net rate of mass outflow in the $x$ direction:
$$
\frac{\partial}{\partial x}\left(\rho u\right) \delta x \delta y \delta z
$$

Nothing speacial about the $x$ direction, hence one can see the net rate of mass outflow can be expressed as,
$$
\left[ 
\frac{\partial}{\partial x}\left(\rho u\right) + \frac{\partial}{\partial y}\left(\rho v\right) + \frac{\partial}{\partial z}\left(\rho w\right)
\right]
\delta x \delta y \delta z
$$

From which we finally get the differential form:
$$
\frac{\partial \rho}{\partial t} 
+ \frac{\partial}{\partial x}\left(\rho u\right) + \frac{\partial}{\partial y}\left(\rho v\right) + \frac{\partial}{\partial z}\left(\rho w\right)
= 0
$$
In a nicer form,
$$
\frac{\partial \rho}{\partial t} + \nabla \cdot \rho\mathbf{v} = 0
$$

Notice that for a steady flow (constant fluid velocity) the first term will be zero.




\subsection{Conservation of Momentum}
$$
\mathbf{F} 
= \frac{D}{D t} \int \mathbf{V} dm 
$$

Following the same aproach as before can obtain the differential form, which has an unsteady and a flux term, 
$$
\Sigma f
= \frac{\partial}{\partial t} \int_{cv} \mathbf{v} \rho \, dV
+ \int_{cs} \mathbf{v} \rho \mathbf{v}\cdot\mathbf{\hat{n}} \, dA
$$

Here again, we need to step back, look at our control volume and gather some inpiration so see what terms we need to take into account.

For an infitesimal volume of fluid, we will have,
$$
\delta \mathbf{f} = \frac{D}{D t}\left(\mathbf{v} \delta m\right)
= \delta m \cdot \mathbf{a}
$$

The system will experience two types of forces, body and surface forces.
For our case, the only body force we may take into consideration may be that of gravity, in which case $\delta f_b = \delta m \cdot \mathbf{g}$.

Surface forces will consist of, think if a cube, one normal to the surface, and two tangential perpendicular to the normal force and to each other.
From thsi point of view we will go on and define stresses:

\begin{itemize}
\item Normal stress: 
$$\sigma_n = \lim_{\delta A \to 0} \frac{\delta f_b}{\delta A}$$

\item Shearing stress:
$$
\tau_i = \lim_{\delta A \to 0} \frac{\delta f_i}{\delta A}
$$
\end{itemize}

Now, stresses make sense when you taken into account with respect to a coordinate system.
Using plain and simple cartetian we will define the following notation:
For the stresses on a plane in direction of coordinate $i$ and parallel to the other two, we will talk about the normal stress $\sigma_{ii}$ and $\tau_{ij}$ and $\tau_{ik}$ (assuming that the directions $j$ and $k$ are orthogonal to direction of the normal vector, $i$, for that plane).
So first subscript is the direction of the normal vector, second subscript is the direction of the stress vector.

Sign convention? The positive direction for the stress is the positive coordinate direction on the surface for which the outward normal is in the positive coordinate direction.


Looking at the surface forces in terms of stresses, starting with planes orthogonal to the $y-z$ directions, we then have, for the net normal stress,
$$
\left[
	\left( 
	\sigma_{xx} 
	+ \frac{\partial \sigma_{xx}}{\partial x} \frac{\delta x}{2}
	\right)
	-
	\left( 
	\sigma_{xx} 
	- \frac{\partial \sigma_{xx}}{\partial x} \frac{\delta x}{2}
	\right)
\right] \delta y \delta z 
= \frac{\partial \sigma_{xx}}{\partial x} \delta x \delta y \delta z
$$

The net shear stresses, same configuration,
$$
\left[
	\left(
	\tau_{zx} 
	+ \frac{\partial \tau_{zx}}{\partial z} \frac{\partial z}{2}
	\right)
	-
	\left(
	\tau_{zx} 
	- \frac{\partial \tau_{zx}}{\partial z} \frac{\partial z}{2}
	\right)
\right] \delta x\delta y
= \frac{\partial \tau_{zx}}{\partial z} \delta x\delta y\delta z
$$ 
and
$$
\left[
	\left(
	\tau_{yx} 
	+ \frac{\partial \tau_{yx}}{\partial y} \frac{\partial y}{2}
	\right)
	-
	\left(
	\tau_{yx} 
	- \frac{\partial \tau_{yx}}{\partial y} \frac{\partial y}{2}
	\right)
\right] \delta x\delta z
= \frac{\partial \tau_{yx}}{\partial y} \delta x\delta y\delta z
$$ 

Here, once again, we are fishing for patterns. Let's look at the total forces in the $x$ direction due to surface forces:
$$
\delta f_{sx} 
= 
\left(
	\frac{\partial \sigma_{xx}}{\partial x} + \frac{\partial \tau_{yx}}{\partial y} + \frac{\partial \tau_{zx}}{\partial z} 
\right) \delta x\delta y\delta z
$$

Smilarly,
$$
\delta f_{sy} 
= 
\left(
	\frac{\partial \tau_{xy}}{\partial x} + \frac{\partial \sigma_{yy}}{\partial y} + \frac{\partial \tau_{zy}}{\partial y} 
\right) \delta x\delta y\delta z
$$

and,
$$
\delta f_{sz} 
= 
\left(
	\frac{\partial \tau_{xz}}{\partial x} + \frac{\partial \tau_{yz}}{\partial y} + \frac{\partial \sigma_{zz}}{\partial z} 
\right) \delta x\delta y\delta z
$$

\textbf{Equations of Motion:}
Given that $\delta f = \delta m \cdot \mathbf{a}$ and $\delta m = \rho \delta x\delta y\delta z$. 
We can obtain equations of motion by noting that the sum of body and surface forces should equal the sum unsteady and convective acceleration terms.
 
$$
\rho g_x 
+ 
\left(
	\frac{\partial \sigma_{xx}}{\partial x} + \frac{\partial \tau_{yx}}{\partial y} + \frac{\partial \tau_{zx}}{\partial z} 
\right)
= \rho 
\left( 
	\frac{\partial u}{\partial t} 
+ u \frac{\partial u}{\partial x}
+ v \frac{\partial u}{\partial y}
+ w \frac{\partial u}{\partial z}
\right)
$$

$$
\rho g_x 
+ 
\left(
	\frac{\partial \tau_{xy}}{\partial x} + \frac{\partial \sigma_{yy}}{\partial y} + \frac{\partial \tau_{zy}}{\partial y} 
\right)
= \rho 
\left( 
	\frac{\partial v}{\partial t} 
+ u \frac{\partial v}{\partial x}
+ v \frac{\partial v}{\partial y}
+ w \frac{\partial v}{\partial z}
\right)
$$

$$
\rho g_x 
+ 
\left(
	\frac{\partial \tau_{xz}}{\partial x} + \frac{\partial \tau_{yz}}{\partial y} + \frac{\partial \sigma_{zz}}{\partial z}
\right)
= \rho 
\left( 
	\frac{\partial v}{\partial t} 
+ u \frac{\partial w}{\partial x}
+ v \frac{\partial w}{\partial y}
+ w \frac{\partial w}{\partial z}
\right)
$$


\end{document}